A partir de los objetivos principales del gobierno de Mar Chiquita:
\begin{itemize}
\item Aliviar la red de transporte público.
\item Mejorar la salud de los habitantes.
\item Disminuír la cantidad de autos particulares en tránsito. 
\end{itemize}

Construimos el siguiente diagrama refinando gradualmente cada uno de ellos. El diagrama se lee de la siguiente manera:
Cada nodo representa un objetivo a cumplir, y de él se desprenden otros que intentan responder al interrogante de cómo cumplirlos. En otras palabras, los nodos inferiores contribuyen a que se cumplan los superiores.
Debido a la gran extensión del diagrama decidimos separarlo en varias partes, y para cada una de ellas detallaremos los 
nodos más importantes.

\subsection{Root}

Esta porción del diagrama es la raíz del mismo. Aquí aparecen los 3 objetivos principales del gobierno de Mar Chiquita.
Como se puede notar, estimular el uso de la red de ciclovías es el principal objetivo que contribuye a lograrlos.
Nuestro trabajo consiste en abordar los refinamientos de este nodo.

\subsection{Estimular el uso de la red}

Consideramos que este objetivo se cumple cuando la red funciona de manera eficiente (buen manejo de los
alquileres) y se mantiene en buen estado. Además, es necesario que sea conocida por la mayor cantida de gente
posible y poder actuar en función de las opiniones y sugerencias de los usuarios.

\subsection{Alquiler root}

En nuestra opinión, este objetivo se logra cumpliendo principalmente con 3 sub objetivos:
Administrar eficazmente los usuarios, entregando bicis de manera eficaz y recibiendo bicicletas de igual
manera. 
Junto con ellos pero en menor medida, consideramos que agregar la posibilidad de que los usuarios puedan realizar consultas
(por ejemplo, conocer la disponibilidad de stock en las distintas estaciones en todo momento) contribuye a cumplir con el mismo.

\subsection{Administración de usuarios eficiente}

Administrar los usuarios implica básicamente registrar a las personas que utilizarán el sistema para obtener sus datos y 
poder identificarlas posteriormente. Al mismo tiempo es necesario aplicar penalizaciones correspondientes a aquellas que 
utilizen indebidamente la red, llevando un registro de estos hechos que pueda ser consultado por los administradores. 

\subsection{Proceso de registración de usuario simple}

Para disminuir la complejidad durante el proceso de registración, proponemos ofrecer asistencia personalizada, especialmente
para aquellas personas que tengan dificultades a la hora de utilizar nuevas tecnologías. Para ello planteamos diversas
alternativas: Brindar asistencia en estaciones, en la oficina central o telefónica.






