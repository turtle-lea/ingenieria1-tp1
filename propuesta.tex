\subsection{Propuesta a los stakeholders}

El gobierno de Mar Chiquita atraviesa un colapso general del transporte público y al mismo tiempo, un incremento reciente en la cantidad de automóviles particulares en circulación. También se ha puesto en evidencia el problema de salud general que atraviesa de Mar Chiquita, ya que estadísticas recientes muestran un alto índice de obesidad y sedentarismo de la gente de la localidad.

Se nos ha comunicado que una estrategia a seguir para aliviar estas problemáticas es estimular el uso de la nueva Red de Ciclovías de Mar Chiquita.

Modificar la cotidianeidad de las personas es un objetivo difícil de lograr, pero confiamos en que la realización de un plan de acción completo y de alcance general que concientize a la población, mejore la infraestructura y que facilite el uso de la Red podrá lograr dicho objetivo.

Con ese objetivo en mente, “Grupo 4”, elaboró una propuesta para llevar a cabo la estrategia planteada.

El relevamiento de los requerimientos y objetivos, ha puesto en evidencia la necesidad de proporcionar un sistema informático que permita resolver la disponibilidad de bicicletas, principalmente en las horas pico, facilitar el retiro de las bicicletas, permitir a los usuarios acceder a servicios de la Red mediante Internet para el registro y chequeo del estado de la red y disponibilidad de las bicicletas en las estaciones de la Red y que la administración a su vez pueda ser realizada mediante un sistema centralizado a través de Internet. 

Nuestra propuesta consta de distintas mecánicas que, instrumentadas en conjunto, favorecerán la adopción de la Red a la vida cotidiana de las personas y lograrán que la población de Mar Chiquita acceda a un sistema de Red de Ciclovías gratuito, confiable, efectivo y de nivel internacional.
