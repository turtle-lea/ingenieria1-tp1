Nuestro prestigioso equipo de desarrolladores de software llamado ''Grupo 4'' ha sido solicitado por el gobierno de
Mar Chiquita para el desarrollo de un sistema que administre su nueva red de ciclovías, con el objetivo de acercar
a los habitantes a este nuevo medio de transporte y lograr estimular su uso.

Los empleados del gobierno nos han compartido sus expectativas en que las ciclovías contribuyan a aliviar la 
red de transporte público y disminuir la cantidad de automóviles particulares en circulación. Al mismo tiempo
creen que el uso de las bicicletas es una buena manera intentar reducir el sobrepeso que sufre la mayor parte 
de la población, al atacar una de las principales causas como lo es el sedentarismo. 

Al mismo tiempo, algunos futuros usuarios nos han hecho llegar sus comentarios e inquietudes respecto al uso que
desean darle a la red: Disponibilidad en las estaciones del centro durante las horas pico, preocupación respecto
a la caída de conexión de internet en alguna de las estaciones, reposición de bicicletas cuando se agota el stock,
buena disponibilidad de las mismas, protección e integridad de los datos de los usuarios, entre otros. 
Los empleados del gobierno que supervisarán el sistema también quieren contar con información relevante (estadísticas,
comentarios de los usuarios, etc) que les permitan tomar mejores decisiones a futuro.

Considerando todos estos puntos, planificamos y diseñamos una serie de documentos a través de los cuales pretendemos 
llegar a un acuerdo con los stakeholders respecto a los requerimientos que debe cumplir nuestro sistema.

Para ello, debemos asegurarnos
de haber comprendido correctamente los objetivos que se esperan de este software, y aquellos que han motivado la
creación del mismo. La creación de un diagrama de objetivos nos pareció una buena herramienta para establecer
una base concreta sobre la cual trabajar.

Al mismo tiempo debemos ponernos de acuerdo respecto al alcance de nuestro software.
Una buena aproximación es, en primer lugar, determinar el conjunto de todos los eventos que se relacionan con 
la red de ciclovías y el sistema administrador a desarrollar, distinguiendo entre aquellos que se corresponden 
exclusivamente con la red, aquellos que lo hacen exclusivamente con el sistema y finalmente, aquellos 
eventos que relacionan ambas componentes. Para ello decidimos utilizar el modelo de Jackson.
Luego, utilizando los eventos relevantes, diseñamos un diagrama de contexto mostrando
las interacciones entre los distintos agentes, y entre estos y el sistema a desarrollar.