\vspace{1cm}
\textbf{\underline{W: World}}

\begin{itemize}
\item Usuario va a buscar una bicicleta a una estación de la periferia.
\item Usuario va a buscar una bicicleta a una estación del centro.
\item Usuario retira una bicicleta de alguna estación
\item Usuario devuelve una bicicleta en alguna estación
\item Usuario se identifica en una estación, presentando DNI.
\item Usuario se registra en el sistema.
\item Usuario se desplaza entre estación y estación.
\item Los camiones de la empresa de transporte proveen una estación con más bicicletas.
\item Los camiones de la empresa de transporte retiran bicicletas de una estación.
\item Un empleado de gobierno provee a la empresa de transporte de mayor cantidad de bicicletas.
\item Empleado de estación rechaza solicitud de retiro de bicicleta debido a penalización.
\item Una estación no dispone de bicicletas para retirar
\item Empleado de gobierno inaugura estación añadiéndola a la red de ciclovías.
\item Empleado de gobierno contrata persona para que trabaje como empleado de estación.
\item Usuario nunca devuelve una bicicleta
\item Usuario (que ya dispone de una) solicita otra bicicleta.
\item Alguien se hace pasar por un usuario para retirar/devolver una bicicleta
\item La gente se traslada ida y vuelta a su trabajo/casa en una franja horaria reducida, llamada ''hora pico''
\item Usuario sufre un accidente.
\item La bicicleta sufre un desperfecto.
\item Usuario rompe una bicicleta
\item Usuario no cumple con las normativas para el uso de la red de ciclovías.
\item Usuario olvida contraseña
\item Usuario espera por bicicleta
\item Usuario devuelve una bicicleta que no está en condiciones.
\item Usuario devuelve una bicicleta fuera de horario.
\item Usuario devuelve una bicicleta en una estación cerrada
\item Ninguna estación dispone de bicicletas
\item Un nuevo usuario no provee la información necesaria para inscribirse.
\item Empleado le entrega una bicicleta a un usuario que no está registrado.
\item Empleado le entrega una bicicleta a un usuario que está penalizado.
\item Gobierno entrega más bicicletas a la empresa de transportes.
\item Sistema pierde la conexión.
\item La empresa de transporte no cumple con el tiempo estimado en trasladar las bicicletas.
\item Una estación queda aislada
\item La empresa de transporte no se puede movilizar.
\item Muchos usuarios se registran a la vez.
\end{itemize}

\vspace{1cm}
\textbf{\underline{I: Interface}}

\begin{itemize}

\item Empleado de estación envía (via internet) información de autenticación de un usuario para ser validada por el sistema.
\item Empleado de estación ingresa (via internet) en el sistema modificación de stock de bicicletas.
\item Empleado de estación ingresa (via internet) en el sistema penalización a usuario.
\item Empleado de estación ingresa (via internet) información de bicicleta en mal estado.
\item Empleado de estación envía información de autenticación de un usuario para ser validada por el sistema via SMS, debido
a la pérdida de conexión de internet de la estación en donde trabaja.
\item Empleado de estación ingresa en el sistema modificación de stock de bicicletas via SMS, debido
a la pérdida de conexión de internet de la estación en donde trabaja.
\item Empleado de estación ingresa en el sistema penalización a usuario via SMS, debido
a la pérdida de conexión de internet de la estación en donde trabaja.
\item Empleado de estación ingresa información de bicicleta en mal estado via SMS, debido
a la pérdida de conexión de internet de la estación en donde trabaja.


\item Empleado de gobierno realiza consulta de estadísticas relevantes al sistema.
\item Empleado de gobierno realiza consulta sugerencias de usuarios al sistema.
\item Empleado de gobierno ingresa nueva estación al sistema.

\item Sistema solicita a movilización de bicicletas a la empresa de transporte.
\item Sistema devuelve al empleado de estación el resultado de autenticación válida o inválida
\item Sistema envía a usuario información de stock de bicicletas disponibles para una determinada estación.
\item Sistema envía a usuario tiempo restante para que haya una bicicleta disponible en una determinada estación, en el caso
de que no haya stock.
\item Sistema envía a usuario información acerca de sus penalizaciones.

\item Usuario se registra en el sistema.
\item Usuario ingresa sugerencia en el sistema.
\item Usuario consulta stock de bicicletas en una determinada estación.
\item Usuario consulta acerca de sus penalizaciones.

\end{itemize}


\vspace{1cm}
\textbf{\underline{M: Machine}}

\begin{itemize}
\item El sistema calcula el mejor conjunto de estaciones que disponen de bicicletas para otra estación.
\item El sistema detecta que una estación se ha quedado sin bicicletas.
\item El sistema detecta que en una estación sobran bicicletas, al contrastar la cantidad actual con los datos
de las estadísticas recientes.
\item El sistema actualiza el estado de un usuario.
\item El sistema detecta una discrepancia entre la solicitud de un usuario y el estado del usuario en el sistema.
\item El sistema registra estadísticas relevantes
\item El sistema actualiza el estado de una estación.
\item El sistema recalcula distribución de bicicletas cuando se agrega una estación a la red.
\end{itemize}


