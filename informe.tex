\documentclass[a4paper,11pt]{article}
\usepackage[utf8]{inputenc}
\usepackage[paper=a4paper, hmargin=1.5cm, bottom=1.5cm, top=3.5cm]{geometry}
\usepackage[T1]{fontenc}
\usepackage[spanish]{babel}
\usepackage[colorlinks=true, linkcolor=blue]{hyperref} %Links para el indice.
\usepackage{amsfonts}
\usepackage{verbatim}
\usepackage{listings}
\usepackage{caption}
\usepackage{subcaption}

\usepackage[section]{placeins}
\usepackage{float}
%\usepackage{adjustbox}
\usepackage{amsmath}
\usepackage{blindtext}
\usepackage{sidecap}
\usepackage{color}

% \newcommand{\real}{\hbox{\bf R}}

\title{Trabajo Práctico de Ingeniería de Software I}

\begin{document}

\maketitle

\begin{center}
	Universidad de Buenos Aires - Departamento de Computaci\'on - FCEN
\end{center}

\vspace{2cm}
Integrantes:

\begin{itemize}
	\item Castro, Dami\'an L.U.: 326/11  \verb+ltdicai@gmail.com+
	\item Matayoshi, Leandro L.U.: 79/11 \verb+leandro.matayoshi@gmail.com+
	\item Melnik, Jonathan L.U.: 571/09 \verb+jonathanmelnik@gmail.com+
	\item Santos, Martín L.U.: 413/11 \verb+martin.n.santos@gmail.com+
	\item Szyrej, Alexander L.U.: 642/11   \verb+alexanderszyrej@gmail.com+
	
\end{itemize}

\newpage

\tableofcontents

\newpage

\section{Eventos del modelo de Jackson}

Con la intención de concretizar los requerimientos de la propuesta utilizamos el ampliamente conocido modelo de Jackson: 
\footnote{\url{http://users.mct.open.ac.uk/mj665/icse17kn.pdf}}

Se basa en estudiar los requerimientos, separando los fenómenos que pertenecen al Mundo(World), la Interfaz y el Sistema(Machine). Esta distinción permite exhibir el propósito del sistema, el cual se encuentra en el Mundo y delimitar el alcance
del mismo. 

Hicimos un relevamiento, y a continuación exponemos un listado con todos los fenómenos(eventos o situaciones observables) que encontramos.
Los fenómenos pertenecientes al Mundo serán objetivos relevados y las presunciones que elaboramos y los que pertenecen a la Interfaz son los requerimientos del sistema que responden a los eventos en el Mundo.


\vspace{1cm}
\textbf{\underline{W: World}}

\begin{itemize}
\item Usuario va a buscar una bicicleta a una estación de la periferia.
\item Usuario va a buscar una bicicleta a una estación del centro.
\item Usuario retira una bicicleta de alguna estación
\item Usuario devuelve una bicicleta en alguna estación
\item Usuario se identifica en una estación, presentando DNI.
\item Usuario se desplaza entre estación y estación.
\item Los camiones de la empresa de transporte proveen una estación con más bicicletas.
\item Los camiones de la empresa de transporte retiran bicicletas de una estación.
\item Un empleado de gobierno provee a la empresa de transporte de mayor cantidad de bicicletas.
\item Empleado de estación rechaza solicitud de retiro de bicicleta debido a penalización.
\item Una estación no dispone de bicicletas para retirar
\item Empleado de gobierno inaugura estación añadiéndola a la red de ciclovías.
\item Empleado de gobierno contrata persona para que trabaje como empleado de estación.
\item Usuario nunca devuelve una bicicleta
\item Usuario (que ya dispone de una) solicita otra bicicleta.
\item Alguien se hace pasar por un usuario para retirar/devolver una bicicleta
\item La gente se traslada ida y vuelta a su trabajo/casa en una franja horaria reducida, llamada ''hora pico''
\item Usuario sufre un accidente.
\item La bicicleta sufre un desperfecto.
\item Usuario rompe una bicicleta
\item Usuario no cumple con las normativas para el uso de la red de ciclovías.
\item Usuario olvida contraseña
\item Usuario espera por bicicleta
\item Usuario devuelve una bicicleta que no está en condiciones.
\item Usuario devuelve una bicicleta fuera de horario.
\item Usuario devuelve una bicicleta en una estación cerrada
\item Ninguna estación dispone de bicicletas
\item Un nuevo usuario no provee la información necesaria para inscribirse.
\item Empleado le entrega una bicicleta a un usuario que no está registrado.
\item Empleado le entrega una bicicleta a un usuario que está penalizado.
\item Gobierno entrega más bicicletas a la empresa de transportes.
\item Sistema pierde la conexión.
\item La empresa de transporte no cumple con el tiempo estimado en trasladar las bicicletas.
\item Una estación queda aislada
\item La empresa de transporte no se puede movilizar.
\item Muchos usuarios se registran a la vez.
\end{itemize}

\vspace{1cm}
\textbf{\underline{I: Interface}}

\begin{itemize}

\item Empleado de estación envía (via internet) información de autenticación de un usuario para ser validada por el sistema.
\item Empleado de estación ingresa (via internet) en el sistema modificación de stock de bicicletas.
\item Empleado de estación ingresa (via internet) en el sistema penalización a usuario.
\item Empleado de estación ingresa (via internet) información de bicicleta en mal estado.
\item Empleado de estación envía información de autenticación de un usuario para ser validada por el sistema via SMS, debido
a la pérdida de conexión de internet de la estación en donde trabaja.
\item Empleado de estación ingresa en el sistema modificación de stock de bicicletas via SMS, debido
a la pérdida de conexión de internet de la estación en donde trabaja.
\item Empleado de estación ingresa en el sistema penalización a usuario via SMS, debido
a la pérdida de conexión de internet de la estación en donde trabaja.
\item Empleado de estación ingresa información de bicicleta en mal estado via SMS, debido
a la pérdida de conexión de internet de la estación en donde trabaja.
\item Empleado de estación consulta datos de usuario: si existe alquiler actual, si el usuario está penalizado
actualmente.


\item Empleado de gobierno realiza cualquiera de las acciones que puede realizar el empleado de estación (vía internet
o SMS). Por ejemlo, empleado de gobierno ingresa modificación de stock.
\item Empleado de gobierno realiza consulta de estadísticas relevantes al sistema.
\item Empleado de gobierno realiza consulta sugerencias de usuarios al sistema.
\item Empleado de gobierno ingresa nueva estación al sistema.
\item Empleado de gobierno ingresa nuevo empleado en el sistema.
\item Empleado de gobierno da de baja empleado en el sistema.
\item Empleado de gobierno actualiza estado de la empresa de transporte

\item Sistema solicita a movilización de bicicletas a la empresa de transporte.
\item Sistema devuelve al empleado de estación el resultado de autenticación válida o inválida
\item Sistema envía a usuario información de stock de bicicletas disponibles para una determinada estación.
\item Sistema envía a usuario tiempo restante para que haya una bicicleta disponible en una determinada estación, en el caso
de que no haya stock.
\item Sistema envía a usuario información acerca de sus penalizaciones.

\item Usuario se registra en el sistema.
\item Usuario se loguea en el sistema.
\item Usuario ingresa sugerencia en el sistema.
\item Usuario completa encuesta respecto a un nuevo lugar en donde agregar una nueva estación.
\item Usuario ingresa comentarios respecto al cambio en su estado de salud desde que es usuario de la red.
\item Usuario consulta stock de bicicletas en una determinada estación.
\item Usuario consulta acerca de sus penalizaciones.

\end{itemize}


\vspace{1cm}
\textbf{\underline{M: Machine}}

\begin{itemize}
\item El sistema calcula el mejor conjunto de estaciones que disponen de bicicletas para otra estación.
\item El sistema detecta que una estación se ha quedado sin bicicletas.
\item El sistema detecta que en una estación sobran bicicletas, al contrastar la cantidad actual con los datos
de las estadísticas recientes.
\item El sistema actualiza el estado de un usuario.
\item El sistema detecta una discrepancia entre la solicitud de un usuario y el estado del usuario en el sistema.
\item El sistema registra estadísticas relevantes
\item El sistema actualiza el estado de una estación.
\item El sistema recalcula distribución de bicicletas cuando se agrega una estación a la red.
\end{itemize}




\section{Escenarios}

\subsection{Algunos escenarios posibles}
\begin{itemize}

\item Un señor cansado de sufrir el tránsito vehicular de Mar Chiquita observa un anuncio en donde se informan los beneficios de usar la nueva red de ciclovías que ofrece el Gobierno. Decide darle una oportunidad. Ingresa al sitio web que vio en el anuncio y comienza a llenar el formulario de inscripción para utilizar el servicio. 

Una vez registrado, el individuo se acerca a la estación más cercana a su hogar. Se encuentra con un empleado que le solicita el DNI. El usuario se lo da y el empleado realiza ciertos chequeos en la computadora de la estación. Verifica que se encuentra todo en orden y procede a darle una bicicleta. El usuario la utiliza y la devuelve en otra estación pasado el tiempo reglamentario. El empleado le notifica que cometió una infracción y le aplica una penalización que le prohibe utilizar bicicletas por una semana. El usuario se retira de la estación. Luego de 3 días el usuario vuelve a acercarse a una estación para retirar una bicicleta. Para ello le entrega su DNI al empleado. Este le informa que se encuentra penalizado y que debe esperar 4 días más para retirar una bicicleta. El usuario se retira.

\item Juan es empleado de una estación del centro de la red de ciclovías. Comienza su jornada laboral. Durante el transcurso del día se acercan varios usuarios con la intención de utlizar bicicletas para lo cual Juan debe realizar registros y autenticaciones de los mismos. Pasan las horas y Juan observa que las bicicletas comienzan a escasear. Luego de unos minutos arriba un camión lleno de bicicletas. Juan actualiza el stock de bicicletas presentes en la estación.
Llega la hora pico y Juan observa que llegan nuevos empleados a la estación. Al rato el caudal de usuarios que se aproxima comienza a aumentar. Aún así el tiempo de espera sigue siendo reducido. En cierto momento se pierde la conexión a internet y Juan comienza a utlizar el servicio de mensajeria de su celular para que el servicio pueda seguir funcionando. Termina la jornada laboral.

\end{itemize}


\end{document}
